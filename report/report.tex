\documentclass[12pt]{article}
\usepackage[utf8]{inputenc}
\usepackage{cleveref}
\usepackage{graphicx}
\usepackage{subfig}
\usepackage{geometry}
\usepackage{bbm}
\usepackage{amsmath}
\usepackage{systeme}
\geometry{a4paper, left=30mm, right=30mm}

\setlength\parindent{0pt}

\newcommand{\R}{\mathbbm{R}}
\newcommand{\1}{\mathbbm{1}}
\title{Conjugate Gradient for \\ Quadratic Minimum-Cost Flow}
\author{Lorenzo Beretta, \texttt{loribere@gmail.com}
  \and Project NoML-13 of C.M. course, C.S. department, UniPi}
\date{3rd July 2019}

\begin{document}
\maketitle


\section{Introduction}
The aim of this project is to exploit the Conjugate Gradient method to solve the
linear system

\begin{equation} \label{linear_system}
  \left(E D^{-1} E^t\right) x = b
\end{equation}

where $E \in \R^{n \times m}$  is the node-edge matrix of a directed graph
and $D \in {\R^{m \times m}$ is a diagonal positive definite matrix.

This problem arises from the KKT conditions of a quadratic separable Minimum-Cost
Flow problem when we do not impose any capacity constraint on arcs.
  
\section{MCF Problem}
Let us derive the equivalence of uncapacited quadratic separable MCF and the linear
system problem stated above.

\subsection{Problem Statement}
Given a directed graph $G = \left(N, A\right)$ such that $|N| = n$ and $|A| = m$ and
a balance vector $b \in \R^n$ such that $\1^t b = 0$ encoding how much
flow is supplied or requested by each node, we want to find a flow vector
$f \in \R^m$ that satisfies $ E f = b $ and minimizes $\frac{1}{2}f^t D f + q^t f$
given a diagonal matrix $D \in \R^{m \times m}$, $D \succeq 0$ and $q \in \R^m$.

\subsection{Optimality as a linear problem}
We can easily reconduct the previous objective function to the simpler
$ \frac{1}{2}\tilde{f}^t D \tilde{f} $, in fact it is sufficient to impose
$\tilde{f} = f - D^{-1} q$, $\tilde{b} = b - E D^{-1} q$ and neglect the
constant offset in the objective function. Then forgetting tildes our
optimization problem becomes:

\begin{equation}
  \begin{gathered}
    \min_{f \in \R^m} f^t D f \\
    E f = b
  \end{gathered}
\end{equation}

Then imposing Lagrangian multipliers conditions (a.k.a. ``poorman'' KKT) we get

\begin{equation*}
  f \text{ is optimal} \iff
  \left\{\begin{gathered}
    f^t D = x^t E \\  
    E f = b
  \end{gatheres}
  \iff
   \left\{\begin{gathered}
    f = D^{-1} E^t x\\  
    \left(E D^{-1} E^t\right) x = b
  \end{gatheres}
\end{equation*}

From now on we can forget the optimization problem and deal just with the linear system.



ESEMPIO DI CITAIZONE
\cite{trefethen97}
  
\bibliographystyle{acm}  
\bibliography{biblio}
\end{document}

%%% Local Variables:
%%% mode: latex
%%% TeX-master: t
%%% End:
